\documentclass[12pt,a4paper,leqno]{amsart}
\usepackage[utf8]{inputenc}  
\usepackage[T1]{fontenc}     
\usepackage[finnish]{babel}   
\usepackage{amsfonts, amsthm}     
\usepackage{amsmath}
\usepackage{amssymb}
\usepackage{ dsfont }
\usepackage{algpseudocode}
\usepackage{algorithm}
\pagestyle{plain}
 
\begin{document}
\begin{flushleft}
				Toteutusdokumentti, Pistepeli
				\end{flushleft}				\begin{flushright}
				Eeva Nikkari
				\end{flushright}
				
				
Algoritmin toimintaidea pseudokoodina:
\begin{algorithm}
\begin{algorithmic}
\Function{topoAlgoritmi}{G}
	\State pino = Topological-Sort(G)
	\While{pino NOT EMPTY}
		\State u = pino.pop()
			\ForAll{v in vierus[u]} 
				\State \Call{Relax}{u,v}
			\EndFor
	\EndWhile
\EndFunction
\\

\Function{Relax}{u,v} 
\If{pisteSaalis[v] < pisteSaalis[u] + v.pisteArvo}
\State pisteSaalis[v] = pisteSaalis[u] + v.pisteArvo
\State reitti[v] = u
\EndIf
\EndFunction
\end{algorithmic}
\end{algorithm}


Algoritmi kutsuu ensin verkolle G Topological-Sort(G) -algoritmia, jonka aikavaativuus on tunnetusti O(|V|+|E|), missä V on solmujen- ja E kaarien joukko. Sitten While-looppi käy läpi kerran jokaisen solmun verkossa eli suorittaa itsensä |V| kertaa. Loopissa käydään yhteensä kerran läpi jokainen solmu ja jokainen kaari. Relax-metodi on aikavaativuudeltaan vakio. Koko algoritmin aikavaativuudeksi tulee siis O(2(|V|+|E|))=\textbf{O(|V|+|E|)}. Käytännössä algoritmi vielä lopuksi etsii parhaan pistemäärän taulukosta pisteSaalis ja tulkitsee parhaan reitin taulukosta reitti. Nämä toiminnot ovat kuitenkin lineaarisia solmujen määrän suhteen, joten aikavaativuusarvio säilyy samana.\\

Algoritmi käyttää aputaulukkoja pisteSaalis, reitti sekä pinoa. Lisäksi Topological-Sort -algoritmi käyttää aputaulukkoa kayty. Kaikkien taulukkojen ja pinon tilavaativuus on O(|V|). Koko algoritmin tilavaativuus on siis O(4|V|)=\textbf{O(|V|)}.\\

Algoritmin eräs puutos on, että se ei tunnista, jos sille annetaan verkko, jossa on sykli. Algoritmi ei siis toimi oikein, jos verkossa on sykli. Käytännössä ongelmia tulee, kun algoritmi yrittää tulkita reitti-taulukkoa ja algoritmi jää looppiin.



\end{document}
 